


\documentclass[a4paper, norsk, 8pt]{article}
\usepackage[utf8]{inputenc}
\usepackage[T1]{fontenc}
\usepackage{babel, textcomp, color, amsmath, amssymb, tikz, subfig, float,esint}
\usepackage{amsfonts}
\usepackage{graphicx}
\usepackage{multicol}
\usepackage{tikz}
\usepackage{pgfplots}

\newcommand{\EQU}[1] { \begin{equation*} \begin{split}
#1  
\end{split} \end{equation*} }
 \newcommand{\DE}[1] {  \begin{description}  #1 \end{description} }
 \newcommand{\IT}[2] { \item[\color{blue} #1]{#2} }
 \newcommand{\vv}[1] { \mathbf{#1} }
 \newcommand{\PAR}[2]{ \frac{\partial #1}{\partial #2}}
 \newcommand{\expe}[1] { \left\langle#1\right\rangle }
 \newcommand{\ket}[1] { |#1\rangle }
  \newcommand{\bra}[1] { \langle #1 | }
  \newcommand{\braket}[2] { \langle #1 | #2 \rangle }
  \newcommand{\commutator}[2]{ \left[ #1 , #2\right] }
  \newcommand{\colvec}[2] { 
  \left( \begin{matrix}
 #1 \\
 #2 \\
  \end{matrix}\right) }
 \newcommand{\PLOTS}[4]{ 
\begin{tikzpicture}
\begin{axis}[
    axis lines = #3, %usally left
    xlabel = #1,
    ylabel = #2,
]
#4
\end{axis}
\end{tikzpicture}
}


\newcommand{\addPLOT}[4]{
\addplot [domain=#1:#2,samples=200,color=#3,]{#4};}
\newcommand{\addCOORDS}[1]{\addplot coordinates {#1};}
\newcommand{\addDRAW}[1]{\draw #1;}
\newcommand{\addNODE}[2]{ \node at (#1) {#2};}

%		\PLOTS{x}{y}{left}{
%			\ADDPLOT{x^2}{-2}{2}{blue}
%			\ADDCOORDS{(0,1)(1,1)(1,2)}
%		}




\definecolor{svar}{RGB}{0,0,0}
\definecolor{opgavetekst}{RGB}{109,109,109}
\definecolor{blygraa}{RGB}{44,52,59}

\hoffset = -60pt
\voffset = -95pt
\oddsidemargin = 0pt
\topmargin = 0pt
\textheight = 0.97\paperheight
\textwidth = 0.97\paperwidth

\begin{document}
\tiny
\begin{multicols*}{3}
\subsubsection*{\scriptsize Enheter og størrelser i FYS2140}
Avogradros tall: $N_A=6.023\times 10^{23}$ \\
Elektronvolt: $1 eV = 1.602\times10^{-19}J$ \\
$\hbar = 6.582\times10^{-16}eVs$ \\
Bohrradien: $a_0=\frac{\hbar^2}{2m_eke^2}\approx0.0529$nm \\
$hc\approx1240$eVnm og $\hbar c\approx197.35$eVnm. \\
Elektronmassen: $m_e\approx 0.511$MeV \\
Protonmassen: $m_p\approx 938.3$MeV \\
Nøytronmassen: $m_n\approx 939.6$MeV \\
Atommasseenheten: $u\approx 931.5$MeV \\
Coulombfaktor: $ke^2=1.44$eVnm

\subsection*{\footnotesize  EKSPERIMENTER}


\subsubsection*{\scriptsize Bruddet med klassisk fysikk}
\begin{tabular}{ l l l}
1898 & Marie Curie & Radioaktiv polonium og radium \\
1900 & Planck      & Plancks kvantiseringshypotese \\
     &             & og sort legeme-stråling \\
1905 & Einstein    & Fotoelektrisk effekt \\  
1911 & Rutherford  & Atommodell \\
1913 & Bohr        & Kvanteteori for atomspektra \\
1922 & Compton     & Spredning av fotoner på elektroner \\
1923 & Goudsmit    & Elektronets egenspinn \\
     & og Uhlenbeck&  \\
1924 & Pauli       & Paulis eksklusjonsprinsipp \\
1925 & De Broglie  & Materiebølger \\
1926 & Schrödinger & Bølgelikning og ny naturlov \\
1927 & Heisenberg  & Uskarphetsrelasjonen \\
1927 & Davidsson   & Eksperiment som påviste  \\
     & og Germer   & materiens bølgeegenskaper  \\
1927 & Born        & Tolkningen av bølgefunksjonen \\
1928 & Dirac       & Relativistisk kvantemekanikk \\
     &             & og prediksjon av positronet  \\
\end{tabular}


\subsubsection*{\scriptsize Sort legeme-stråling og Plancks kvantiseringshypotese}
Introduserer ideen om kvantiserte av energier. Et sort legeme kan bare emittere bestemte kvanta med energi.
$M(T)=\sigma T^4$, \ $\lambda_{\tiny \mbox{max}}T=2.897\times10^{-3}$Km. \textit{Enhver fysisk størrelse som utviser harmoniske svingninger har energier som tilfredsstiller $E_n(\nu)=nh\nu$, der $\nu$ er frekvensen og $h$ er en universell konstant ($n\in \mathbb{Z}^+$).}


\subsubsection*{\scriptsize Fotoelektrisk effekt} 
Lys/elektromegnetisk stråling sendes mot en metallplate. Elektroner kastes løs. Elektronene trekkes mot anoden(positiv). Det finnes en øvre kinetisk energi $K_{\tiny \mbox{max}}=eV_0$ som gjør at ingen elektroner kommer til anoden. $\bullet$ $K_{\tiny \mbox{max}}$ er uavhengig av lysintensiteten $I$. $\bullet$ Det finnes en laveste frekvens ($K_{\tiny \mbox{max}}$ øker lineært med frekvens). $\bullet$ Elektronene løsrives umiddelbart.\\
$\omega_0$: arbeidsfunksjon (arbeidet som må gjøres for å rive løs de svakest bundne elektronene). $K_{\tiny \mbox{max}}=h\nu-\omega_0$.


\subsubsection*{\scriptsize Röntgenstråling} 
Det motsatte av fotoelektrisk effekt: Elektroner aksellereres gjennom et potensialfall $V_R$ (noen tusen volt), og treffer et metall. Oppbremsingen av elektronene fører til emisjon av röntgenstråling. Kinetisk energi $K_e=eV_R$ og hvis elektronet stopper helt: $h\nu=K_e$ der $\nu$ er maksimal frekvens. \underline{Brudd med klassisk:} Det finnes en minste bølgelengde $\lambda_{\tiny \mbox{min}}=hc/eV_R$ for utsending av Röntgenstråling (en maksimal energi for de utsendte fotonene).


\subsubsection*{\scriptsize Comptonspredning} 
Fotoner kan tilordnes bevegelsesmengde. Sendte høyenergetiske fotoner mot en grafittplate og observerte at bølgelengden til den ''spredte'', utgående strålen vår større enn bølgelengden til den innkommende. Konstruktiv interferens (vha. Bragg diffraksjon) $n\lambda=2d\sin \phi$.
$E=\sqrt{p^2c^2+m^2c^4}=pc=h\nu \Rightarrow p=h/\lambda$. Bevaring av energi og bevegelsesmengde i kollisjonen foton $\rightarrow$ elektron gir $\Delta \lambda = \Lambda_C(1-\cos \theta)$ med $\lambda_C=h/m_e c = 2.43\times 10^{-3}$nm.


\subsubsection*{\scriptsize Bohrs atommodell} 
\textbf{Bohrs postulater:} 
$\bullet$ Et elektron beveger seg i sirkulær baner om en kjerne. Kreftene er gitt ved Coulombkraften $F_e = \frac{1}{4\pi\epsilon_0}\frac{e^2}{r}$.
$\bullet$ Bare visse elektronbaner er stabile. I disse banene sender elektronene \textit{ikke} ut e.m. stråling.
$\bullet$ Et diskontinuerlig ''hopp'' fra en bane med energi $E_i$ til en bane med energi $E_f$ fører til at det sendes ut stråling med frekvens $\nu=\frac{1}{h}(E_i-E_f)$. Dersom $E_i<E_f$ absorberes stråling med frekvens $-\nu$. 
$\bullet$ De tillatte elektronbanene er bestemt ved at deres angulærmoment $L$ rundt kjernen er kvantisert: $L=mvr=n\hbar$ der $n$ er et heltall.
\textbf{Resultater:} Bohr finner \textit{Bohrradien:} $a_0=-\frac{k^2m_ee^4}{2\hbar^2}\frac{1}{n^2}\approx -13.6\mbox{eV}\frac{1}{n^2}$. og Bruker dette til å reprodusere \textit{Rydbergs formel} for strålingsoverganger mellom ulike energinivåer: $\frac{1}{\lambda}=\frac{ke^2}{2a_0hc}\left(\frac{1}{n_f^2}-\frac{1}{n_i^2}\right)$.


\subsubsection*{\scriptsize Davidsson-Germer eksperimentet og Bragg diffraksjon}
\textbf{Davidsson-Germer:} Likner dobbelspalteeksperimentet. En elektronbølge sendes mot et et krystallgitter. Hvert atom fungerer som en punktkilde. Da registreres interferensmønster med konstruktiv interferens for $n\lambda = d \sin \theta$.
\textbf{Bragg diffaksjon:} Energirike elektroner sendes med innfallsvinkel $\theta$ mot et krystallgitter (da trenger de lengre inn). En detektor plasseres med samme vinkel på andre side. Strålen fra atomlag 2 reiser $2d\sin\theta$ lengre enn strålen som reflekteres fra overflaten. Konstruktiv interferens blir da gitt av $2d\sin \theta = m\lambda$ (\textit{Braggs lov}).


\subsubsection*{\scriptsize De Broglies hypotese}
Partikler med endelig masse utviser både partikkel og bølgeegenskaper. De Broglie foreslo at all materie har en bølgelengde $\lambda = h/p$. Materiebølger må da også ha en frekvens $\nu=E/h$. For partikler med masse er $\nu \lambda $ \textit{ikke} lik partiklens hastiget, slik det er for lys. De broglie forslo to fundamentale relasjoner \[p=\frac{h}{\lambda}=\hbar k \mbox{ og } E=h\nu=\hbar \omega.\]


\subsubsection*{\scriptsize Franck Hertz eksperimentet} 
Bekreftelse på eksistensen til stasjonære tilstander i atomer (kun bestemte eksiterte tilstander var tillatt). Elektroner aksellereres av et potensial $V_{\mbox{KG}}$ gjennom en kvikksølvgass (fra katode til gitter). Deretter bremses elektronene (fra gitter til anode) av en spenning $V_{\mbox{GA}}$ med $V_{\mbox{GA}}<V_{\mbox{KG}}$. De elektronene som ikke kommer frem har kollidert med kvikksølvatomer. Økende spenningsforskjell $V_{\mbox{KA}}$ får flere elektroner gjennom, men ikke linær sammenheng. Elektronstrøm som funksjon av potensialforskjell viste topper ved heltallsmultipler av $4.9$V. Når kvikksølvatomene deeksiteres sender de ut lys med $\lambda=254$nm. Det gir $\Delta E =hc/\lambda \approx 4.88$eV, som er de målte verdiene.

\subsubsection*{\scriptsize Zeemaneffekten}
Når et atom plasseres i et konstant magnetfelt forskyves energinivåene av hamilton-leddet $H_Z=-\frac{e}{2m}(\mathbf{L}+2\mathbf{S})\cdot\mathbf{B}_{\mbox{ext}}$, der $\mathbf{S}$ er elektronets spinn, og $\mathbf{L}$ er angulærmomentet. \textbf{Den anomale Zeemaneffekten:} Spinn $\mathbf{S}$ tas med i beregningen. Merk: \textit{Elektronets innebygde dipolmoment er gitt ved} $\mathbf{\mu}_s=-g_e\frac{e}{2m_e}\mathbf{S}$ der $g_e$ er elektronets gyromagnetiske faktor $\approx 2$. \textbf{Den normale Zeemaneffekten:} Bare angulærmoment $\mathbf{L}$ tas med: $H_Z=-\frac{e}{2m}\mathbf{L}\cdot\mathbf{B}_{\mbox{ext}}$. Egentilstandene for den utvidede hamiltonoperatoren blir de samme ($R(r)Y_{lm}(\theta,\phi)$) men energien blir $E_{nm}=-E_0/n^2+eBm\hbar/2m_e$ med $E_0\approx13.6$eV.

\subsubsection*{\scriptsize Stern-Gerlach eksperimentet}
Skulle undersøke Bohrs hypotese om kvantisert angulærmoment $L$. Sølvatomer (med ett ytre elektron: fungerer derfor som hydrogenatomer) sendes gjennom et inhomogent magnetfelt. Vi fant kvantifiseringen $L=\hbar\sqrt{\ell(\ell+1)},L_z=m\hbar$ der $\ell=0,1,2,...$ og $m=-l,...,l$. Dipolmoment avhenger av angulærmoment $\mathbf{\mu}=-\frac{e}{2m_e}\mathbf{L}$ og kraften er gitt ved $\mathbf{F}=\nabla (\mathbf{\mu}\cdot\mathbf{B})$. Splittingen av sølvatomstrålen er dermed kvantisert. Hydrogenatomer i (1s)-tilstanden ville gitt én linje, men det finnes spinn. Selv om $l=0$ finnes $\uparrow$ og $\downarrow$ som splitter strålen i to.

\subsubsection*{\scriptsize Spinn-bane-kobling}
I elektronets referansesystem sirkulerer protonet rundt og skaper et konstant magnetfelt. Dette er opphav til det korrigerende hamilton-leddet $H=-\mathbf{\mu}\cdot\mathbf{B}=-\mathbf{\mu}\cdot \frac{1}{4\pi \epsilon_0}\frac{e}{mc^2 r^3}\mathbf{L}$. Dette ''reduserer'' degenerasjonen litt og skaper en riktigere modell for atomer. Da er ikke lenger $\ket{n\ell m_\ell m_s}$ egenfunksjon til Hamiltonoperatoren.

\subsubsection*{\scriptsize Gruppe-og fasehastighet}
\textit{For en planbølgeløsning:}
Gruppehastighet $v_g = \frac{\mbox{d}\omega}{\mbox{d}k}=\frac{\mbox{d}}{\mbox{d}k}\frac{\hbar k^2}{2m}=\frac{\hbar k}{m}=\frac{p}{m}=v$ er partikkelens fysiske hastighet. Fasehastighet $v_f = \frac{\omega}{k} = \frac{1}{2}v_g$ er bølgetoppenes hastighet.  

\subsection*{\footnotesize KVANTEMEKANIKK}
\subsubsection*{\scriptsize Operatorer} 
\begin{tabular}{|l|c|l|}
\hline
Posisjon: 				& $\hat{x}$		& $x$ \\ 
Bevegelsesmengde: 		& $\hat{p}$		& $-i\hbar \frac{\partial}{\partial x}=m\frac{d\expe{x}}{dt}$ 			\\ 
Kinetisk energi:	 	& $\hat{K}$		& $-\frac{\hbar^2}{2m}\nabla^2$ 										\\ 
Hamilton:				& $\hat{H}$		& $\hat{K}+\hat{V}$ 													\\ 
Energi:					& $\hat{E}$		& $i\hbar \PAR{}{t}$ 													\\ 
Angulærmoment:			& $\hat{L}_z$	& $-i\hbar\left(x\PAR{}{y}-y\PAR{}{x}\right)=-i\hbar\PAR{}{\phi}$ 		\\
HO Stigeoperator:		& $\hat{a}_\pm$	& $\frac{1}{\sqrt{2\hbar m \omega}}\left(\pm i\hat{p}+m\omega\hat{x}\right)$ 		\\ \hline
\end{tabular}

\subsubsection*{\scriptsize Kommutatorer}
Inkompatible operatorer: $\commutator{\hat{A}}{\hat{B}}=\hat{A}\hat{B}-\hat{B}\hat{A}\neq 0$\\
\begin{tabular}{|l|l|}
\hline
$\commutator{\hat{p}}{\hat{x}}=i\hbar$				& $\commutator{\hat{L^2}}{\hat{L}_z}=0$ 	\\
$\commutator{\hat{L}_x}{\hat{L}_y}=i\hbar L_z$		& $\commutator{\hat{E}}{\hat{t}}=i\hbar$ 	\\
$\commutator{\hat{a}_-}{\hat{a}_+}=1$				& $\commutator{\hat{E}}{\hat{x}}=0$   		\\ \hline
\end{tabular} 


\subsubsection*{\scriptsize Uskarphetsrelasjonen og Ehrenfests teorem} 
$\expe{G}=\int_{\mbox{\tiny all space}}\Psi^*G\Psi \mbox{d}(\mbox{space})$, $\sigma_G=\sqrt{\expe{G^2}-\expe{G}^2}$.\\
\textbf{Uskarphetsrelasjonen:} $\sigma_A^2 \sigma_B^2 \geq \left(\frac{1}{2i}\commutator{\hat{A}}{\hat{B}}\right)^2$.
$\sigma_x\sigma_p \geq \frac{\hbar}{2}$\\
\textbf{Ehrenfests teorem:} Dersom $\Phi$ er en kvantemekanisk tilstand får vi $i\hbar\PAR{\Phi}{t}=\hat{H}\Phi$. Dette gir
$\frac{\mbox{d}}{\mbox{d}t}\expe{A} =\frac{\mbox{d}}{\mbox{d}t} \bra{\Phi}\hat{A}\ket{\Phi} = \int_{-\infty}^{\infty} \PAR{\Phi^*}{t}\hat{A}\Phi\mbox{d}x + \int_{-\infty}^{\infty} \Phi^*\PAR{\hat{A}}{t}\Phi\mbox{d}x+\int_{-\infty}^{\infty} \Phi^*\hat{A}\PAR{\Phi}{t}\mbox{d}x = \int_{-\infty}^{\infty} \left[\left(-\frac{1}{i\hbar}\Phi^*H^*\right)\hat{A}\Phi+\Phi^*\hat{A}\left(\frac{1}{i\hbar}\hat{H}\Phi\right) \right]\mbox{d}x+\expe{\PAR{\hat{A}}{t}}$ som siden $H^*=H$ gir \[
\frac{\mbox{d}}{\mbox{d}t}\expe{A}=\frac{1}{i\hbar}\expe{\commutator{\hat{A}}{\hat{H}}}+\expe{\PAR{\hat{A}}{t}}.\]

\subsubsection*{\scriptsize TASL 3D}
Den tidsavhengige Schrödingerlikningen \[i\hbar \frac{\partial \Psi}{\partial t} = -\frac{\hbar^2}{2m}\nabla^2 \Psi+V(\vec{r})\Psi\]  

\subsubsection*{\scriptsize TUSL 1D}
Separasjon av variable: $\Psi(x,t)=\psi(x)\phi(t)$, med $\phi(t)=e^{-\frac{i}{\hbar}Et}$, $\psi(x)$ gitt av
\[\PAR{^2\psi}{x^2}=\frac{2m}{\hbar^2}(V-E)\psi\]
Løsningene av TUSL tilfredsstiller: \\
$\bullet$ De er \textit{Stasjonære tilstander}: $\Psi(x,t)=\psi(x)e^{-iEt/\hbar} \Rightarrow \frac{\mbox{d}}{\mbox{dt}}|\Psi|^2=\frac{\mbox{d}}{\mbox{dt}}|\psi|^2=0$. Dette gir også $\expe{Q}=\bra{\Psi}\hat{Q}\ket{\Psi}=\int \Psi^* Q(x,\frac{\hbar}{i}\frac{\mbox{d}}{\mbox{dx}})\Psi\mbox{dx}=\int \psi^* Q(x,\frac{\hbar}{i}\frac{\mbox{d}}{\mbox{dx}})\psi\mbox{dx}$ \\
$\bullet$ De har skarpt bestemt energi. Det vil si $\sigma_H^2=\expe{H^2}-\expe{H}^2=E^2-E^2=0$. \\
$\bullet$ De danner et komplett sett. Den generelle løsningen er en lineærkombinasjon av de stasjonære tilstandene:
\[ \Psi(x,t)=\sum_{n=1}^{\infty}c_n\psi_n(x)e^{-iE_nt/\hbar} \]

\subsubsection*{\scriptsize Sfæriske koordinater }
$x=r \cos{\phi} \sin \theta,y=r\sin \phi \sin \theta,z=r\cos\theta,dxdydz=r^2\sin \theta drd\theta d\phi$, der $r\in[0,\infty],\phi\in[0,2\pi]$ og $\theta\in[0,\pi]$
\[
\nabla^2 = \frac{1}{r^2}\PAR{}{r}\left(r^2\PAR{}{r}\right)+\frac{1}{r^2\sin\theta}\PAR{}{\theta}\left(\sin \theta \PAR{}{\theta}\right)+\frac{1}{r^2\sin^2 \theta} \PAR{^2}{\phi^2}
\] 
eller, om ønskelig
\[ \nabla^2 = \PAR{^2}{r^2}+\frac{2}{r}\PAR{}{r}-\frac{\hat{L}^2}{\hbar^2r^2} \]


\subsubsection*{\scriptsize Postulater}
\begin{itemize}
\item Tilstanden til et system bestemmes av en bølgefunksjon $\Psi(x,t)$ som er en løsning av Shrödingerlikningen $\hat{H}\Psi=i\hbar \PAR{}{t}\Psi$
\item Til enhver observabel $Q$ svarer det en lineær, hermitisk operator $\hat{Q}$
\item De eneste mulige resultatene av en måling av observabelen $Q$ er en av egenverdiene $q_n$ til operatoren $\hat{Q}$. $\hat{Q}\ket{\Psi_n}=q_n\ket{\Psi_n}$ der $\ket{\Psi_n}$ er en egenfunksjon.
\item Et ensemble av systemer preparert i tilstanden $\Psi$ vil ha en forventningsverdi for en observabel $Q$ lik $\expe{Q}=\bra{\Psi}\hat{Q}\ket{\Psi}$
\item Hvis $\hat{Q}$ er en hermitisk operator og $\{\Psi_n\}$ er settet av alle egentilstander til $\hat{Q}$, så er dette settet komplett. Dvs en vilkårlig tilstand $\Psi$ kan skrives $\Psi = \sum_n c_n \Psi_n$
\end{itemize}


\subsubsection*{\scriptsize Forhold mellom $\Psi$ og $V$}
$\bullet$ Hvis $V=\infty$ må $\Psi=0$. $\bullet$ I områder der $V\neq \infty$ er både $\Psi$ og $\PAR{\Psi}{x}$ kontinuerlig.


\subsubsection*{\scriptsize Uendelig brønn}
Potensialet $V(x)=0$ når $0\leq x \leq a$ og $V(x)=\infty$ ellers. Utenfor må vi ha $\psi(x)=0$ og innenfor 
\[ \frac{\mbox{d}^2\psi}{\mbox{dx}^2}=-k^2\psi \mbox{ der } k\equiv \frac{\sqrt{2mE}}{\hbar}.  \]
Det gir $\psi=A\sin{kx}+B\cos{kx}$. Kontinuitetskravet gir $\psi(0)=\psi(a)=0$ som betyr at $B=0$ og $A\sin{ka}=0$. Det kvantifiserer $k$ ved $k_n=\frac{n\pi}{a},n\in\mathbb{N}$.\\
\textbf{Når et potensial går mot uendelig kreves ikke lenger kontinuitet av $\frac{d\psi}{dx}$.}\\
Fra definisjonen av $k$ fører kvantifiseringen til kvantiserte energinivåer:
\[ E_n=\frac{\hbar^2k_n^2}{2m}=\frac{n^2\pi\hbar^2}{2ma^2}, \ \psi_n(x)=\sqrt{\frac{2}{a}}\sin\left(\frac{n\pi}{a}x\right). \]
Løsningene er $\bullet$ annenhver odd og jevn med hensyn med brønnens midtpunkt, $\bullet$ Antall nullpunkter øker med én per økt $n$, $\bullet$ de er orthogonale og $\bullet$ de er komplette. 

 
\subsubsection*{\scriptsize Harmonisk oscillator} 
HO potensial: $V = \frac{1}{2}m\omega^2 x^2$ gir HO TUSL
\[ -\frac{\hbar^2}{2m}\frac{\mbox{d}^2\psi}{\mbox{d}x^2}+\frac{1}{2}m\omega^2x^2\psi=E\psi \]
med $E_n=\left(n+1/2\right)\hbar \omega $ og $\psi_n(x)=\frac{1}{\sqrt{n!}} (\hat{a}_+)^n \psi_0(x)$ der $\psi_0(x)=\left(\frac{m\omega}{\pi \hbar}\right)^{\frac{1}{4}}e^{-\frac{m\omega}{2\hbar}x^2}$ og $\hat{a}_+ \psi_{n}= \sqrt{n+1}\psi_{n+1}$ med $n=0,1,2,...$ og $\hat{a}_- \psi_{n}= \sqrt{n}\psi_{n-1}$. \\
Algebraisk løsning fremkommer av å skrive hamiltonfunksjonen som \[\hat{H}=\frac{1}{2m}\left[\hat{p}^2+(m\omega \hat{x})^2\right]=\hbar \omega \left[\hat{a}_-\hat{a}_+-\frac{1}{2}\right].\]
Videre må $\hat{H}\ket{\Psi_n}=E_n\ket{\Psi_n}\Rightarrow\hat{H}\ket{\hat{a}_\pm \Psi_n}=E_{n\pm 1}\ket{\Psi_{n\pm 1}}$ og siden $\hat{a}_-$ senker energien må det, for å unngå negative energier, finnes $\hat{a}_-\Psi_0=0$. \\
Første tre løsninger, med $\alpha=m\omega/\hbar$:\\
\begin{tabular}{ |l|}
\hline
$\psi_0 = \left(\frac{\alpha}{\pi}\right)^\frac{1}{4}e^{-\frac{\alpha}{2}x^2}$ \\ \hline
$\psi_1 = \sqrt{2\alpha}\left(\frac{\alpha}{\pi}\right)^\frac{1}{4}xe^{-\frac{\alpha}{2}x^2}$  \\ \hline
$\psi_2 = \left(\sqrt{2}\alpha^\frac{3}{2}x^3-\frac{3}{\sqrt{2}\sqrt{\alpha}}x \right)\left(\frac{\alpha}{\pi}\right)^\frac{1}{4}e^{-\frac{\alpha}{2}x^2}$ \\ \hline
\end{tabular}

For flerdimensjonal harmonisk oscillator kan man bruke separasjonen $\Psi(x_1,x_2,...,x_n,t)=\Psi(x_1,t)\Psi(x_2,t)...\Psi(x_n,t)$ med energier $E=E_{x_1}+E_{x_2}+...+E_{x_n}$.


\subsubsection*{\scriptsize Fri partikkel}
For en fri partikkel er $V(x)=0$. Det gir den generelle løsningen på TUSL
\[ \Psi(x,t)=Ae^{ikx-i\frac{\hbar k^2}{2m}t}+Be^{-ikx-i\frac{\hbar k^2}{2m}t} \]
der $k>0$: mot høyre og $k<0$: mot venstre. Løsningen er altså \textbf{ikke normaliserbar}.
Det har ikke forekommet noen kvantisering, da blir rekkeutviklingen vha. stasjonære tilstander til et integral der $c_n\mapsto \phi(k)dk$ som betyr at en helt generell løsning gir
\[ \Psi(x,t)=\frac{1}{\sqrt{2\pi}}\int_{-\infty}^\infty \phi(k)e^{i(kx-\frac{\hbar k^2}{2m}t)}\mbox{dk} \]
med $\phi(k)=\frac{1}{\sqrt{2\pi}} \int_{\mathbb{R}}\Psi(x,0)e^{-ikx}\mbox{dx}$.


\subsubsection*{\scriptsize Endelig brønn}
Gitt potensialet $V(x)=-V_0\mathbb{I}_{\{-a\leq x \leq a\} }(x)$ kan man definere $\kappa\equiv\frac{\sqrt{-2mE}}{\hbar}$. For $x\leq a$ er $Ae^{-\kappa x}$ en ufysisk løsning (går mot $\infty$ når $x$ går mot $-\infty$). For $x\geq a$ er $Be^{\kappa x}$ ufysisk. Inne i brønnen kan vi definere $l\equiv \frac{\sqrt{2m(E+V_0)}}{\hbar}$ og den generelle løsningen $\psi_M = C\sin{lx}+D\cos{lx}$. Løsningen er altså $\psi_L(x) = Fe^{-\kappa x}, \psi_M(x) = D\cos{lx}$ og $\psi_R(x) = \psi_L(-x)$. Kontinuitet av $\psi$ og $\frac{\mbox{d}\psi}{\mbox{dx}}$ i $a$ gir $Fe^{-\kappa a}=D\cos{la}$ og $-\kappa Fe^{-\kappa a}=-lD\sin{la}$, som kan slås sammen til $\kappa=l\tan{la}$. Ved å skrive $z\equiv la$ og $z_0\equiv \frac{a}{\hbar}\sqrt{2mV_0}$ får vi $\tan z = \sqrt{(z_0/z)^2-1}$

\subsubsection*{\scriptsize Kvantemekanikk i 3D}
Schrödingerlikningen i tre dimensjoner tar formen \[-\frac{\hbar^2}{2m}\nabla^2\Psi(\mathbf{r},t)+V\Psi(\mathbf{r},t)=\hbar{H}\Psi(\mathbf{r},t)=\hat{E}\Psi(\mathbf{r},t)=i\hbar \PAR{\Psi(\mathbf{r},t)}{t}.\]
Dersom man bruker separasjonen $\Psi(\mathbf{r},t)=\psi(\mathbf{r})\phi(t)$, får $\phi(t)$ de vanlige løsningene $\exp{(-iEt/\hbar)}$ og $\psi$ løses av en egenverdilikning $\hat{H}\psi=E\psi$. Dersom potensialet $V(\mathbf{r})$ har sfærisk symmetri, kan man benytte separasjonen $\psi(\mathbf{r})=R(r)Y(\theta,\phi)$. Dette gir to differensiallikninger der bare den ene, \textbf{Radiallikningen}, avhenger av potensialet $V$. Den andre, \textbf{Angulærlikningen}, kan splittes videre ved å la $Y(\theta,\phi)=\Theta(\theta)\Phi(\phi)$. Den generelle løsningen blir da $\Theta(\theta)=AP_\ell^m(\cos \theta)$ der $P_\ell^m(x) \equiv (1-x^2)^{\frac{|m|}{2}}\frac{\mbox{d}^{|m|}}{\mbox{d}x^{|m|}}\frac{1}{2^\ell \ell!}\frac{\mbox{d}^{\ell}}{\mbox{d}x^{\ell}}(x^2-1)^\ell$ er de assosierte Legendre polynomene. Videre er $\Phi(\phi)=e^{im\phi}$. Den fullstendige, normaliserte løsningen av angulærlikningen kalles for de \textbf{sfærisk harmoniske}.

\subsubsection*{\scriptsize Sfæriske harmoniske}
De sfærisk harmoniske kan regnes ut ved \[Y_\ell^m(\theta,\phi)=\epsilon\sqrt{\frac{(2\ell+1)}{4\pi}\frac{(\ell-|m|)!}{(\ell+|m|)!}}e^{im\phi}P_\ell^m(\cos \theta)\] der $\epsilon=(-1)^m$ dersom $m\geq 0$ og $\epsilon=0$ ellers. Videre er$P_\ell^m(x)$ de assosierte Legendre polynomene og funksjonene har egenskapen $\braket{Y_{\ell'}^{m'}}{Y_\ell^m}=\delta_{\ell \ell'}\delta_{mm'}$.\\
\begin{tabular}{|l|}
\hline
$Y_0^0=\left(\frac{1}{4\pi}\right)^\frac{1}{2}$								\\
$Y_1^0=\left(\frac{3}{4\pi}\right)^\frac{1}{2}\cos \theta$ 					\\ 
$Y_1^{\pm1}=\mp \left(\frac{3}{8\pi}\right)^\frac{1}{2}\sin \theta e^{\pm i \phi}$				\\
$Y_2^0=\left(\frac{5}{16\pi}\right)^\frac{1}{2}(3\cos^2\theta-1)$ 				\\ 
$Y_2^{\pm1}=\mp\left(\frac{15}{8\pi}\right)^\frac{1}{2}\sin \theta \cos \theta e^{\pm i \phi}$				\\
$Y_2^{\pm 2}=\left(\frac{15}{32\pi}\right)^\frac{1}{2} \sin^2 \theta e^{\pm 2 i \phi}$ 				\\ 
$Y_3^0=\left(\frac{7}{16\pi}\right)^\frac{1}{2}(5\cos^3 \theta-3\cos \theta)$				\\
$Y_3^{\pm 1}=\mp\left(\frac{21}{64\pi}\right)^\frac{1}{2}\sin\theta(5 \cos^2 \theta-1)e^{\pm i \phi}$ 				\\
$Y_3^{\pm 2}=\left(\frac{105}{32\pi}\right)^\frac{1}{2}\sin^2 \theta \cos \theta e^{\pm 2i \phi}$				\\
$Y_3^{\pm 3}=\mp \left(\frac{35}{64\pi}\right)^\frac{1}{2} \sin^3 \theta e^{\pm 3i\phi}$ 				\\  \hline
\end{tabular} 


\subsubsection*{\scriptsize Radiallikningen}
$\bullet$ Effektivt potensial: $V_{\tiny \mbox{eff}}=V+\frac{\hbar^2}{2m}\frac{\ell(\ell+1)}{r^2}$ 
$\bullet$ Tusl funk. av radius: $-\frac{\hbar^2}{2m}\frac{\mbox{d}^2u}{\mbox{d}r^2}+V_{\tiny \mbox{eff}}u=Eu$ der $u=rR(r)$. 


\subsubsection*{\scriptsize Hydrogenatomet} 
Elektronene blir påvirket av protoner i kjernen ved hjelp av coulombpotensialet $V(\mathbf{r})=V(r)=-\frac{e^2}{4\pi\epsilon_0}\frac{1}{r}$, som settes inn i radiallikningen. Det innføres $\kappa\equiv\frac{1}{\hbar}\sqrt{-2mE},\rho\equiv\kappa r, \rho_0\equiv\frac{me^2}{2\pi\epsilon_0\hbar\kappa}$. Som etter løsning ved rekkeutvikling gir $\rho_0=2n$ og derfor
\[E_n=-\left[\frac{m}{2\hbar^2}\left(\frac{e^2}{4\pi \epsilon_0}\right)^2\right]\frac{1}{n^2}=\frac{E_1}{n^2},n=1,2,3,... \]
med $E_1\approx -13.6$eV. Degenerasjonen er da $d(n)=\sum_{i=0}^{n-1}=n^2$. Løsningen av TUSL blir da 
\[ \psi_{nlm}=A_{n\ell}e^{-r/na}\left(\frac{2r}{na}\right)^\ell \left[L_{n-\ell-1}^{2\ell+1}\left(\frac{2r}{na}\right)\right]Y_\ell^m(\theta,\phi) \] der $A_{n\ell}=\sqrt{\left(\frac{2}{na}\right)^3\frac{(n-\ell-1)!}{2n[(n+\ell)]^3}}$, $a\equiv\frac{4\pi\epsilon_0\hbar^2}{me^2}\approx 0.529 \times 10^{-10}m$ og \[L_{q-p}^{p}(x)\equiv (-1)^p\left(\frac{\mbox{d}}{\mbox{d}x}\right)^p e^x \left(\frac{\mbox{d}}{\mbox{d}x}\right)^q (e^{-x}x^q)\] er de
\textbf{Assossierte Laguerre polynomene:}\\
\begin{tabular}{|l|l|}
\hline
$L_0^0(x)=1$				& 
$L_0^2(x)=2$ 				\\ 
$L_1^0(x)=-x+1$  			&
$L_1^2(x)=-6x+18$			\\ 
$L_2^0(x)=x^2-4x+2$ 		& 
$L_2^2(x)=12x^2-96x+144$  	\\
$L_0^1(x)=1$				& 
$L_0^3(x)=6$ 				\\ 
$L_1^1(x)=-2x+4$  			&
$L_2^1(x)=3x^2-18x+18$		\\ 
$L_2^3(x)=60x^2-600x+1200$ & \\ \hline
\end{tabular} \\
med hovedkvantetall($E_n=-E_1/n^2$) $n=1,2,3,...$, angulærmomentkvantetall ($L=\pm \hbar\sqrt{\ell(\ell+1)}$) $\ell=0,1,2,...,n-1$ og magnetisk kvantetall ($L_z=\hbar m$) $m=-\ell,-\ell+1,...,-1,0,1,...,\ell-1,\ell$. \\
Energidifferanser $E_\gamma=E_i-E_f=-13.6\mbox{eV}(1/n_i^2-1/n_f^2)$ sammen med $E=h\nu$ gir et analytisk uttrykk for \textbf{Rydberg konstanten} $R\equiv \frac{m}{4\pi c\hbar^3}(e^2/2\pi\epsilon_0)^2\approx 1.097 \times 10^7\mbox{m}^{-1}$, definert ved den empiriske loven (\textbf{Rydberg formelen}): $\frac{1}{\lambda}=R(1/n_f^2-1/n_i^2)$.\\
Dersom spinn tas med blir bølgefunksjonen $\Psi_{n\ell m_\ell} \chi_{m_s}$ der $\Psi$ er romdelen og $\chi$ er spinndelen. Kvantetallet $m_s = -s,-s+1,...,s-1,s$ er gitt av partikkelens spinn $s$, som er $1/2$ for fermioner, $1$ for bosoner og $0$ for Higgs. Med spinn dobles degenerasjonen: $d(n)=2n^2$. Det er med mindre spinn-bane-kobling eller andre justeringer tas med i betraktningen.

\subsubsection*{\scriptsize Spinn} 
Spinn er et ''indre angulærmoment''. Mens angulærmoment kan tenkes på som $\mathbf{L}=\mathbf{r}\times\mathbf{p}$ kan vi tenke på spinn som $\mathbf{S}=I\mathbf{\omega}$. Vi har $\commutator{S_x}{S_y}=i\hbar S_z$ og tilsvarende for $x\mapsto y\mapsto z \mapsto x$. Fra angulærmoment følger det at spinn må tilfredsstille $\hat{S}^2\ket{s,m_s}=\hbar^2s(s+1)\ket{s,m_s}$ og $\hat{S}_z\ket{s,m_s}=\hbar m_s\ket{s,m_s}$, men $s=0,1/2,1,3/2,...$ og $m_s=-s,-s+1,...,s-1,s$. Alle fermioner har uforanderlig $s=1/2$. Da lar vi $\uparrow$ representere $m_s=1/2$ og $\downarrow$ representere $m_s=-1/2$. \\

\subsubsection*{\scriptsize Topartikkelsystemer} 
Topartikkelsystemer kan modeleres ved addisjon av kinetisk energi $\hat{K}=\hat{K}_1+\hat{K}_2$ og et potensial $V(\mathbf{r}_1,\mathbf{r}_1,t)$. Med integralet over sannsynlighetstettet $\int_{\mathbb{R}^3\times \mathbb{R}^3}|\Psi(\mathbf{r}_1,\mathbf{r}_1,t)|^2\mbox{d}^3\mathbf{r}_1\mbox{d}^3\mathbf{r}_2=1$. \\
\textbf{Fermioner og Bosoner:}\\
For to ikke-identiske partikler $1$ og $2$ i tilstand henholdsvis $\psi_a(\mathbf{r}_1)$ og $\psi_b(\mathbf{r}_2)$ er bølgefunksjonen (uten spinn) gitt som et produkt \[\psi(\mathbf{r}_1,\mathbf{r}_2)=\psi_a(\mathbf{r}_1)\psi_b(\mathbf{r}_2).\] Dette stemmer bare for $b\neq a$. \\
For identiske partikler, som f.eks. elektroner eller fotoner, må bølgefunksjonen enten være symmetrisk eller antisymmetrisk om ombytte av partiklene: \[\psi(\mathbf{r}_1,\mathbf{r}_2)=\pm \psi(\mathbf{r}_2,\mathbf{r}_1).\] Her svarer $+$ til Bosoner og $-$ til fermioner. Vi kan da konstruere en bølgefunksjon:
\[ \psi_\pm(\mathbf{r}_1,\mathbf{r}_2)=A\left[\psi_a(\mathbf{r}_1)\psi_b(\mathbf{r}_2)\pm \psi_a(\mathbf{r}_2)\psi_b(\mathbf{r}_1)\right]. \]
Merk at dersom vi ser på fermioner med $a=b$ får vi $\psi_-(\mathbf{r}_1,\mathbf{r}_2)=A\left[\psi_a(\mathbf{r}_1)\psi_a(\mathbf{r}_2)-\psi_a(\mathbf{r}_2)\psi_a(\mathbf{r}_1)\right]=0$. Som er det vi kaller \\ 
\textbf{Paulis eksklusjonsprinsipp:} \textit{ To identiske fermioner kan \underline{aldri} befinne seg i samme tilstand til samme tid}.\\
Når spinn tas med må den totale bølgefunksjonen være antisymmetrisk for fermioner og symmetrisk for bosoner. For spinn-$1/2$ partikler har man total spinn $s=s_1+s_2\in\{0,1\}$ og dermed mulighetene $m_s=m_{s_1}+m_{s_2}\in\{0,\pm 1\}$. Det finnes tre symmetriske muligheter, kalt \textbf{triplet} (s=1):
\[\ket{1 1}=\ket{\uparrow \uparrow},\ket{1 0}=\frac{1}{\sqrt{2}}\left(\ket{\uparrow \downarrow}+\ket{\downarrow\uparrow}\right),\ket{1\mbox{-}1}=\ket{\uparrow \uparrow} \] 
og én antisymmetrisk, såkalt \textit{entangled state}, kalt \textbf{singlett}(s=0):
\[ \ket{00}=\frac{1}{\sqrt{2}}\left(\ket{\uparrow \downarrow}-\ket{\downarrow\uparrow}\right).\]
Koeffisientene $C_{m_1 \, m_2 \, m}$ kalles \textbf{Clebsch-Gordan koeffisienter} og er definert ved
\[\ket{s \, m_s}=\sum_{m_1+m_2=m}C_{m_1 \, m_2 \, m}^{s_1 \, s_2 \, s}\ket{s_1 \, m_1}\ket{s_2 \, m_2}\]eller \[\ket{s_1 \, m_1}\ket{s_2 \, m_2}=\sum_{s}C_{m_1 \, m_2 \, m}^{s_1 \, s_2 \, s}\ket{s \, m_s}.\]
\textbf{''Exchange forces''}\\
For identiske partikler er forventningsverdien til kvadratet av avstanden mellom dem gitt av
\[ \expe{(x_1-x_2)^2}_{\pm}=\expe{x^2}_a+\expe{x^2}_b-2\expe{x}_a\expe{x}_b\mp2|\expe{x}_{ab}|^2 \]
der $\expe{x}_{ab}\equiv \int x \psi_a^* \psi_b \mbox{d} x$. Moralen er: \\
$\bullet$ \textit{Partikler med symmetrisk romdel er nærmere hverandre enn partikler med antisymmetrisk romdel }. \\
For adskillbare partikler mister man $|\expe{x}_{ab}|^2$-leddet og får en mellomting.
\subsubsection*{\scriptsize Grunnstoffer}
Et nøytralt atom med atomnummer $Z$ har Hamiltonfunksjon gitt av
\[ \hat{H}=\sum_{j=1}^{Z}\left\{ -\frac{\hbar}{2m}\nabla^2_j-\left(\frac{1}{4\pi\epsilon_0}\frac{Ze^2}{r_j}\right)\right\}+\frac{1}{2}\left(\frac{1}{4\pi\epsilon_0}\right)\sum_{j\neq k}^{Z}\frac{e^2}{|\mathbf{r}_j-\mathbf{r}_k|}. \]
Den første summen representerer elektronenes kinetiske energi og energien fra kjernens elektriske felt. Andre del representerer energiene fra elektronenes påvirkning av hverandre. Andre del antas lik 0 når det regnes analytisk. \\
\textbf{Kjemikernotasjon:}\\
Kjemikere skriver orbital-tilstander ifølge \\
\centerline{ $(\{n\}\{\ell\})^{\{\# e^- \}}$ } \\
der hovedkvantetallet $n$ og antallet elektroner, $\# e^-$, i tilstanden skrives med tall og det orbitale angulærmomentet $\ell$ med bokstaver: \\
\begin{tabular}{l l}
$s$: ''sharp''  		&$\ell=0$ \\
$p$: ''principal'' 		&$\ell=1$ \\
$d$: ''diffuse'' 		&$\ell=2$ \\
$f$: ''fundamental'' 	&$\ell=3$ \\
\end{tabular}
etterfulgt av $g,h,i,k,l,...$\\
Ved hjelp av \textbf{Hunds regler:}\\
1$\bullet$ \textit{ Så lenge det er konsistent med Paulis prinsipp vil tilstanden med høyest totalt spinn $S$ ha lavest energi.}\\
2$\bullet$ \textit{ For et gitt spinn vil det høyeste totale angulærmomentet $L$, så lenge antisymmetri er overholdt, gi lavest verdi.}\\
3$\bullet$ \textit{ Hvis et skall $(n,\ell)$ er ikke er over halvfullt vil det laveste energinivået ha $J=|L-S|$. Hvis det er mer enn halvfullt må $J=L+S$.}\\
kan totalt spinn skrives på formen $^{2S+1}L_J$:\\
\begin{tabular}{| l l l |}
\hline
H 	& $(1s)$ 				& $^2S_{1/2}$ 		\\ 
He 	& $(1s)^2$				& $^1S_{0}$			\\ 
Li 	& $(1s)^2(2s)$			& $^2S_{1/2}$		\\ 
Be	& $(1s)^2(2s)^2$ 		& $^1S_{0}$			\\ 
B 	& $(1s)^2(2s)^2(2p)$	& $^2P_{1/2}$		\\ 
C 	& $(1s)^2(2s)^2(2p)^2$	& $^3P_{0}$ 		\\ \hline
\end{tabular}

\subsubsection*{\scriptsize Elementærpartikler}
\textbf{Henfall} \\
Ved å redefinere Hamiltonoperatoren $\hat{H}_0$ til 
\[ \hat{H}=\hat{H}_0-\frac{i\hbar}{2}\Gamma \]
er ikke hamiltonoperatoren lenger en hermitisk operator, men sannsynlighetstettheten faller. Dvs sannsynligheten for at partikkelen henfaller øker. 
\[P=\int_{-\infty}^{\infty}|\Psi|^2\mbox{d}x=|\phi(t)|^2\int_{-\infty}^{\infty}|\psi|^2\mbox{d}x=e^{-\Gamma t}\int_{-\infty}^{\infty}|\psi|^2\mbox{d}x. \]
En partikkels \textbf{levetid} er definert ved $\tau \equiv 1/\Gamma$. Når Hamiltonoperatoren er på formen over har ikke engang egentilstandene til $\hat{H}$ skarpt bestemt energi. Siden hvilemassen til en partikkel er gitt ved energien betyr dette at det eksisterer en fundamental uskarphet i en ustabil partikkels masse. Sannsynlighetsfordelingen for massen, $m$, rundt en sentralverdi $m_0$ er gitt av den såkalte \textbf{ikke-relativistiske Breit-Wigner fordelingen}
\[ P(m)=\frac{i\hbar}{2\pi}\frac{1}{(m-m_0)^2+\left(\frac{\hbar \Gamma}{2}\right)^2} \]
\end{multicols*}
\end{document}












